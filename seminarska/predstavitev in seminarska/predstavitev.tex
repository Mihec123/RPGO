\documentclass{beamer}
\usepackage[slovene]{babel}
\usepackage[utf8]{inputenc}
\usepackage[T1]{fontenc}

\usepackage{amsmath}
\usepackage{amsthm}
\usepackage{amsfonts}
\usepackage{url}
\usepackage{graphicx}
\usepackage{enumerate}
\usepackage{listings}
\graphicspath{ {images/} }
 
 
%Information to be included in the title page:
\title{VS postopek za izračun vrednosti polinomov več spremenljivk}
\author{Janez Radešček, Miha Avsec}
\institute{Fakulteta za matematiko in fiziko}
\date{2019}
 
\usetheme{Berlin}
 
 
\begin{document}
 
\frame{\titlepage}

\begin{frame}
\frametitle{Polinom v Bezierjevi obliki}
Naj bo $T$ trikotnik, potem polinom v baricentričnih koordinatah $(r,s,t)$ lahko zapišemo kot
$$p(r,s,t) = \sum_{i=0}^{d}\sum_{j=0}^{i}b_{d-i,i-j,j}B_{d-i,i-j,j}^{d},$$
kjer je
$$B_{i,j,k}^{d}(r,s,t) = \frac{d!}{i!j!k!}r^is^jt^k$$
Bernsteinov polinom stopnje $d$.

\end{frame}

\begin{frame}[fragile]
\frametitle{de Casteljau}
\begin{lstlisting}[escapeinside={(*}{*)}]
for k=1:d
  for i=0:d-k
      for j=0:i
          (*$b_{d-i-k,i-j,j}^k = r*b_{d-i-k+1,i-j,j}^{k-1}+s*b_{d-i-k,i-j+1,j}^{k-1}$*) 
          	+(*$r*b_{d-i-k,i-j,j+1}^{k-1}  $*)
(*$p(r,s,t) = b_{0,0,0}^{d}$*)
\end{lstlisting}


\end{frame}



\begin{frame}
\frametitle{Modificirana Bernstein-Bezierjeva oblika polinoma}
$$p(r,s,t) = \sum_{i=0}^{d}\sum_{j=0}^{i}c_{d-i,i-j,j}r^{d-i}s^{i-j}t^j,$$
kjer za $c_{d-i,i-j,j}$ vzamemo
$$c_{d-i,i-j,j} = \frac{d!}{(d-i)!(i-j)!j!}b_{d-i,i-j,j}, \quad j=0,\ldots, i; i = 0,\ldots,d.$$
\end{frame}


\begin{frame}
\frametitle{Modificirana Bernstein-Bezierjeva oblika polinoma}
Razdelitev domenskega trikotnika v primeru, ko je $d=2$
\begin{center}
\includegraphics[width=.9\linewidth]{graf.png}
\end{center}

\end{frame}

\begin{frame}[fragile]
\frametitle{Modificiran Bernstein-Bezierjev algoritem}
\begin{lstlisting}[escapeinside={(*}{*)}]
sr = s/r,	 tr= s/r
A = (*$c_{0,n,0}$*);
for i = 1:n
    B = (*$c_{0,n-i,i}$*)
    for j = i:-1:1
        B = B*tr + (*$c_{i-j+1,n-i,j-1}$*);
    end
    A = A *sr +B;
end
p(r,s,t) = A(*$r^n$*)
\end{lstlisting}



\end{frame}

\begin{frame}
\frametitle{Taylor}
Zapis polinoma v Taylorjevi obliki
$$p(u,v) = \sum_{i = 0}^n{\sum_{j=0}^{n-i}{a_{i,j}u^iv^j }}$$
\end{frame}



\begin{frame}[fragile]
\frametitle{Taylorjev algoritem}

\begin{lstlisting}[escapeinside={(*}{*)}]
p = (*$a_{0,n}$*)
for i = 1:d
    A = (*$a_{i,n-i}$*)
    for j = 1:i
        A = A * u + (*$a_{i-j,n-i}$*)
    end
    p = p * v + A
end
\end{lstlisting}


\end{frame}

\begin{frame}
\frametitle{Primerjava metod}

\begin{center}
\includegraphics[width=1.0\linewidth]{tabelca.PNG}
\end{center}


\end{frame}

\begin{frame}
\frametitle{Polinom v treh spremenljivkah}
Naj bo T tetraeder v $\mathbb{R}^3$ in naj bodo $(r,s,t,u)$ pripadajoče baricentrične koordinate točke $P$. Potem lahko polinom v točki $P$ zapišemo kot
$$p(r,s,t,u) = \sum_{i=0}^{d}\sum_{j=0}^{i}\sum_{k=0}^{j}c_{d-i,i-j,j-k,k}r^{d-i}s^{i-j}t^{j-k}u^{k}.$$
\end{frame}


\begin{frame}
\frametitle{Polinom v treh spremenljivkah}
\begin{center}
\includegraphics[width=0.7\linewidth]{tetraeder.PNG}
\end{center}
\end{frame}

\begin{frame}[fragile]
\frametitle{Algoritem za polinom v treh spremenljivkah}
\end{frame}































\end{document}